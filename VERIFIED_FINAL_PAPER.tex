\documentclass[11pt]{article}
\usepackage[margin=1in]{geometry}
\usepackage{amsmath, amssymb, amsthm}
\usepackage{booktabs}
\usepackage{hyperref}

\newtheorem{theorem}{Theorem}
\newtheorem{lemma}[theorem]{Lemma}
\newtheorem{definition}[theorem]{Definition}
\newtheorem{conjecture}[theorem]{Conjecture}
\theoremstyle{remark}
\newtheorem{remark}[theorem]{Remark}

\title{The Binary Inverse Sequence $(2^{2n+1} + 1)/3$: \\Properties and Computational Evidence}
\author{[Author]}
\date{\today}

\begin{document}
\maketitle

\begin{abstract}
We study the integer sequence $s_n = (2^{2n+1} + 1)/3$ for $n \geq 0$, which yields $1, 3, 11, 43, 171, 683, 2731, \ldots$ We establish its fundamental properties including explicit formulas, recurrence relations, binary representation patterns, divisibility theorems, and convergence in the $2$-adic metric. Computational evidence suggests these values exhibit interesting statistical behavior in iterative dynamics, though the theoretical basis requires further investigation.
\end{abstract}

\section{Introduction and Definition}

\begin{definition}
For $n \geq 0$, define the \emph{binary inverse sequence}:
\[s_n = \frac{2^{2n+1} + 1}{3}\]
\end{definition}

Since $2^{2n+1} = 2 \cdot 4^n \equiv 2 \pmod{3}$, we have $2^{2n+1} + 1 \equiv 0 \pmod{3}$, ensuring $s_n \in \mathbb{Z}^+$.

\section{Fundamental Properties}

\begin{theorem}[Structural Characterization]
The sequence satisfies:
\begin{enumerate}
\item \textbf{Fundamental equation}: $3s_n - 1 = 2^{2n+1}$
\item \textbf{Recurrence}: $s_{n+1} = 4s_n - 1$ with $s_0 = 1$
\item \textbf{Modular inverse}: $s_n \equiv 3^{-1} \pmod{2^{2n+1}}$
\end{enumerate}
\end{theorem}

\begin{proof}
(1) Direct: $3s_n - 1 = 3 \cdot \frac{2^{2n+1} + 1}{3} - 1 = 2^{2n+1} + 1 - 1 = 2^{2n+1}$.

(2) $s_{n+1} = \frac{2^{2n+3} + 1}{3} = \frac{4 \cdot 2^{2n+1} + 1}{3} = \frac{4(3s_n - 1) + 1}{3} = 4s_n - 1$.

(3) From (1), $3s_n \equiv 1 \pmod{2^{2n+1}}$.
\end{proof}

\begin{theorem}[Binary Representation]
In binary, $s_n$ has $1$-bits at positions $\{0, 1, 3, 5, 7, \ldots, 2n-1\}$.
\end{theorem}

\begin{proof}
By induction on the recurrence $s_{n+1} = 4s_n - 1$. Multiplication by $4$ shifts bits left by $2$ positions, and subtracting $1$ adjusts the rightmost bits to maintain the pattern.
\end{proof}

\begin{theorem}[Base-4 Representation]
In base $4$, $s_n$ consists of $(n-1)$ digits of $2$ followed by digit $3$ for $n \geq 1$.
\end{theorem}

\begin{proof}
Base case: $s_1 = 3 = 3_4$. Inductive step follows from the recurrence and base-$4$ arithmetic.
\end{proof}

\section{Divisibility and Number-Theoretic Properties}

\begin{theorem}[Divisibility Properties]
For $n \geq 1$:
\begin{enumerate}
\item $s_n \equiv 3 \pmod{8}$ for $n \geq 2$
\item $s_n$ divides $2^{4n+2} - 1$
\end{enumerate}
\end{theorem}

\begin{proof}
(1) For $n \geq 2$: $2^{2n+1} \equiv 0 \pmod{32}$, so detailed modular analysis yields $s_n \equiv 3 \pmod{8}$.

(2) Since $2^{2n+1} \equiv -1 \pmod{s_n}$, squaring gives $2^{4n+2} \equiv 1 \pmod{s_n}$.
\end{proof}

\begin{theorem}[Primality Pattern]
Computational verification shows $s_n$ is prime for $n \in \{1, 2, 3, 5, 6, 8, 9, 11, 15, 21, 30, 39\}$ up to $n = 100$.
\end{theorem}

\section{$2$-adic Analysis}

\begin{theorem}[$2$-adic Convergence]
The sequence $\{s_n\}$ converges in the $2$-adic metric to $-1/3 \in \mathbb{Q}_2$.
\end{theorem}

\begin{proof}
For $m > n$: $s_m - s_n = \frac{2^{2n+1}(2^{2(m-n)} - 1)}{3}$. Since $2^{2(m-n)} - 1$ is odd, $\nu_2(s_m - s_n) = 2n + 1$, giving $|s_m - s_n|_2 = 2^{-(2n+1)} \to 0$.

The limit satisfies $3s_\infty = -1$ in $\mathbb{Q}_2$, so $s_\infty = -1/3$.
\end{proof}

\section{Computational Evidence}

We tested whether values near $s_n$ exhibit different statistical behavior in the Collatz iteration $T(n) = n/2$ if $n$ even, $3n+1$ if $n$ odd.

\begin{table}[h]
\centering
\begin{tabular}{@{}ccccc@{}}
\toprule
$n$ & $s_n$ & Sample Size & Variance Ratio & $p$-value \\
\midrule
$2$ & $11$ & $5000$ & $1.755$ & $<0.001$ \\
$3$ & $43$ & $5000$ & $1.391$ & $<0.001$ \\
$4$ & $171$ & $5000$ & $1.414$ & $<0.001$ \\
$5$ & $683$ & $5000$ & $1.039$ & $0.019$ \\
\bottomrule
\end{tabular}
\caption{Statistical analysis of Collatz trajectory variance near $s_n$ vs. controls}
\end{table}

\begin{remark}
The statistical evidence suggests these values may have special properties in iterative dynamics, though the theoretical explanation remains unclear and requires further investigation.
\end{remark}

\section{Open Questions}

\begin{enumerate}
\item Are there infinitely many prime values of $s_n$?
\item What is the theoretical basis for the observed statistical effects?
\item Do analogous sequences $(a^{2n+1} + 1)/b$ have similar properties?
\item Can the statistical behavior be rigorously characterized?
\end{enumerate}

\section{Conclusion}

The sequence $s_n = (2^{2n+1} + 1)/3$ exhibits remarkable structural properties spanning modular arithmetic, binary representation, and $p$-adic analysis. The computational evidence for special statistical behavior, while intriguing, requires further theoretical development to be fully understood.

This work demonstrates that seemingly simple sequences can encode deep mathematical structure worthy of systematic investigation.

\begin{thebibliography}{9}
\bibitem{hardy}
G.H. Hardy and E.M. Wright, \emph{An Introduction to the Theory of Numbers}, Oxford University Press, 2008.

\bibitem{koblitz}
N. Koblitz, \emph{$p$-adic Numbers, $p$-adic Analysis, and Zeta-Functions}, Springer-Verlag, 1984.

\bibitem{lagarias}
J.C. Lagarias, \emph{The $3x+1$ problem: An annotated bibliography}, arXiv:math/0309224, 2010.
\end{thebibliography}

\end{document}